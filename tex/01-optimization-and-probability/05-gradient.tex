\item \points{1e}

Let's practice taking gradients, which is a key operation for being able to
optimize continuous functions. For $\mathbf w \in \mathbb R^d$ (represented as a
column vector) and constants $\mathbf a_i, \mathbf b_j \in \mathbb R^d$ (also
represented as column vectors) and $\lambda \in \mathbb R$, define the
scalar-valued function
$$f(\mathbf w) = \sum_{i=1}^n \sum_{j=1}^n (\mathbf a_i^\top \mathbf w - \mathbf b_j^\top \mathbf w)^2 + \lambda \|\mathbf w\|_2^2,$$
where the vector is $\mathbf w = (w_1, \dots, w_d)^\top$ and
$\|\mathbf w\|_2 = \sqrt{\sum_{k=1}^d w_k^2}$ is known as the $L_2$ norm.
Compute the gradient$\nabla f(\mathbf w)$.

{\em Recall: the gradient is a $d$-dimensional vector of the partial derivatives
with respect to each $w_i$:
$$\nabla f(\mathbf w) = \left(\frac{\partial f(\mathbf w)}{\partial w_1}, \dots \frac{\partial f(\mathbf w)}{\partial w_d}\right)^\top.$$
If you're not comfortable with vector calculus, first warm up by working out
this problem using scalars in place of vectors and derivatives in place of
gradients. Not everything for scalars goes through for vectors, but the two
should at least be consistent with each other (when $d=1$). Do not write out
summation over dimensions, because that gets tedious.}
